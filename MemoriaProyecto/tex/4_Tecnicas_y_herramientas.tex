\capitulo{4}{Técnicas y herramientas}

%Esta parte de la memoria tiene como objetivo presentar las técnicas metodológicas y las herramientas de desarrollo que se han utilizado para llevar a cabo el proyecto. Si se han estudiado diferentes alternativas de metodologías, herramientas, bibliotecas se puede hacer un resumen de los aspectos más destacados de cada alternativa, incluyendo comparativas entre las distintas opciones y una justificación de las elecciones realizadas. 
%No se pretende que este apartado se convierta en un capítulo de un libro dedicado a cada una de las alternativas, sino comentar los aspectos más destacados de cada opción, con un repaso somero a los fundamentos esenciales y referencias bibliográficas para que el lector pueda ampliar su conocimiento sobre el tema.
Este capitulo muestra las herramientas que se han utilizado para la alcanzar los objetivos del proyecto.
\section{Herramientas utilizadas}
\subsection{Entorno de desarrollo}
\begin{description}
	\item[Eclipse IDE for Java EE Developers]. Entorno de programación Java para aplicaciones web. Se ha utilizado la version: 2018-09 (4.9.0).\\ Enlace a página de descarga:\\ \url{https://www.eclipse.org/downloads/}
	\item[Java SE 11 (JDK)]. \textit{Java Development Kit}. Se ha utilizado la versión  v11.0.1.\\ Enlace a página de descarga:\\ \url{https://www.oracle.com/technetwork/java/javase/downloads/index.html}
	\item[Apache Maven]. Gestor de proyectos software que ayuda en la construcción del proyecto, la generación de documentación, generación de informes, gestión de dependencias, integración con un sistema de control de versiones, etc. Se ha utilizado la versión  v3.6.0.\\ Enlace a página de descarga:\\ \url{https://maven.apache.org/download.cgi}
	\item[Apache Tomcat]. Contenedor de aplicaciones web con soporte de servlets Java. Sirve para desplegar la aplicación. Se ha utilizado la versión  v9.0.13.\\ Enlace a página de descarga:\\ \url{https://tomcat.apache.org/download-90.cgi}
\end{description}
\subsection{Logging}
\begin{description}
	\item[SLF4J]. Fachada de logging.\\ Enlace a página de descarga:\\ \url{https://www.slf4j.org/download.html}
	\item[Log4j 2]. Logger. Se ha utilizado la versión  v2.11.2.\\ Enlace a página de descarga:\\ \url{https://logging.apache.org/log4j/2.x/download.html}
\end{description}
\subsection{Pruebas}
\begin{description}
	\item[JUnit5]. Conjunto de bibliotecas para el desarrollo de pruebas unitarias. Se ha utilizado la versión  v5.3.1.\\ Enlace a página de descarga:\\ \url{https://junit.org/junit5/}
\end{description}
\subsection{Frameworks y librerías específicas para el proyecto}
\begin{description}
	\item[gitlab4j-api]. Framework de conexión a GitLab API. Se ha utilizado la versión  v4.9.14.\\ Enlace:\\ \url{https://github.com/gitlab4j/gitlab4j-api}\\
	Se ha preferido frente a timols/java-gitlab-api\footnote{\url{https://github.com/timols/java-gitlab-api}} tras realizar un estudio sobre sus métricas de evolución y una comparativa sobre la documentación. Concluyendo en que \textbf{gitlab4j-api} contiene mejor documentación, mejor evolución y a día de hoy se sigue desarrollando.
	\item[Apache Commons Math]. Librería que se utiliza para matemáticas descriptivas utilizada para el cálculo de cuartiles. Se ha utilizado la versión  v3.6.1.\\ Enlace a página de descarga:\\ \url{https://commons.apache.org/proper/commons-math/download_math.cgi}
\end{description}
\subsection{Interfaz gráfica}
\begin{description}
	\item[Vaadin]. Framework para desarrollo de interfaces web con Java. Se ha utilizado la versión  v13.0.0\\ Enlace:\\ \url{https://vaadin.com/}
\end{description}
\subsection{CI/CD y Calidad de código}
\begin{description}
	\item[GitLab]. Plataforma de desarrollo colaborativo en la que se ha almacenado el proyecto en un repositorio Git.\\ Enlace:\\ \url{https://gitlab.com/users/sign_in}
	\item[Codacy]. Herramienta de generación automática de informes de calidad de código.\\ Enlace:\\ \url{https://www.codacy.com/}
	\item[JaCoCo]. Librería para cobertura de código en Java. Se ha utilizado la versión v0.8.3.\\ Enlace:\\ \url{https://www.eclemma.org/jacoco/}
	\item[Heroku]. Herramienta para despliegue continuo.\\ Enlace:\\ \url{https://id.heroku.com/login}
\end{description}
\subsection{Documentación}
\begin{description}
	\item[LaTeX]. Sistema de composición de textos.\\ Enlace:\\ \url{https://www.latex-project.org/}
	\item[TeXstudio]. Entorno de desarrollo de documentos LaTeX.\\ Enlace:\\ \url{https://www.texstudio.org/}
	\item[Zotero]. Herramienta de gestión de fuentes bibliográficas.\\ Enlace:\\ \url{https://www.zotero.org/}
\end{description}
\subsection{Técnicas}
A lo largo del proyecto se han utilizado numerosos patrones de diseño como Singleton, Factory Method, Wrapper, Builder, Listener, etc.

Para el motor de métricas se ha utilizado como base el framework propuesto en \textit{Soporte de Métricas con Independencia del Lenguaje para la Inferencia de Refactorizaciones} \cite{marticorena_soporte_2005}. Ver figura \ref{fig:MCTMotorMetricas}.
%herramientas codacy maven, gitlab, capturas, etc, cobertura, heroku, ci, pipeline, token...