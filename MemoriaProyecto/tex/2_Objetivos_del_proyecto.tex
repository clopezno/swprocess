\capitulo{2}{Objetivos del proyecto}

%Este apartado explica de forma precisa y concisa cuales son los objetivos que se persiguen con la realización del proyecto. Se puede distinguir entre los objetivos marcados por los requisitos del software a construir y los objetivos de carácter técnico que plantea a la hora de llevar a la práctica el proyecto.
En este capítulo se detallarán los objetivos generales que se desean alcanzar en este proyecto, así como los objetivos más técnicos.

\section{Objetivos generales}
El objetivo general de este TFG es diseñar una aplicación web en Java que permita obtener un conjunto de métricas de evolución del proceso software \cite{ratzinger_space:_2007} reflejado en los repositorios de GitLab, para permitir comparar los distintos procesos de desarrollo software de cada repositorio.
La aplicación se probará con datos reales para comparar los repositorios de Trabajos Fin de Grado del Grado de Ingeniería Informática presentados en GitLab.
   
A continuación se desglosa el objetivo general  en objetivos más detallados.
\begin{itemize}
	\tightlist
	\item Se obtendrán medidas de métricas de evolución de uno o varios proyectos alojados en repositorios de GitLab.
	\item Las métricas que se desean calcular de un repositorio  son algunas de las especificadas en \cite{ratzinger_space:_2007} y 
	adaptadas a los repositorios software:
	\begin{itemize}
		\tightlist
		\item Número total de incidencias (\textit{issues})
		\item Cambios(\textit{commits}) por incidencia
		\item Porcentaje de incidencias cerrados
		\item Media de días en cerrar una incidencia
		\item Media de días entre cambios
		\item Días entre primer y último cambio
		\item Rango de actividad de cambios por mes
		\item Porcentaje de pico de cambios
	\end{itemize}
	\item El objetivo de obtener las métricas es poder evaluar el estado de un proyecto comparándolo con otros proyectos de la misma naturaleza. Para ello se deberán establecer unos valores umbrales de cada métrica basados en el cálculo de los cuartiles Q1 y Q3, respectivamente.
	\item Se desea poder calcular dinámicamente los valores umbrales en perfiles de configuración de medidas. Esta información sobre los
	valores umbrales se puede almacenar de manera persistente para permitir comparaciones futuras. Un ejemplo de utilidad es guardar los valores umbrales 
	de repositorios por lenguaje de programación, o en el caso de repositorios de TFG de la UBU por cursos académico.
\end{itemize}
\section{Objetivos técnicos}
Este apartado recoge los requisitos más técnicos del proyecto.
\begin{itemize}
	\tightlist
	\item Diseñar la aplicación de manera que se puedan extender con nuevas métricas con el menor coste de mantenimiento posible. Se aplicará un diseño basado en frameworks y en patrones de diseño \citep{gamma_patrones_2002}.
	\item El diseño de la aplicación debe facilitar la extensión a otras plataformas de desarrollo colaborativo como GitHub o Bitbucket. Se aplicará un diseño basado en frameworks y en patrones de diseño \citep{gamma_patrones_2002}.
	\item Aplicar el \textit{frameworks} modelo vista controlador para separar la lógica de la aplicación para el calculo de métricas y la interfaz de usuario.
	Se aplicará un diseño basado en frameworks y en patrones de diseño \citep{gamma_patrones_2002}.
	\item Crear una batería de pruebas automáticas con cobertura del por encima del 90\% en los subsistemas de lógica de la aplicación .
	\item Utilizar una plataforma de desarrollo colaborativo que incluya un sistema de control de versiones, un sistema de seguimiento de incidencias y que permita una comunicación fluida entre el tutor y el alumno.
	\item Utilizar un sistema de integración y despliegue continuo.
	\item Diseñar una correcta gestión de errores definiendo excepciones de biblioteca y registrando eventos de error e información en ficheros de \textit{log}. 
	\item Aplicar nuevas estructuras  del lenguaje Java para el desarrollo, como son expresiones lambda. 
	\item Utilizar sistemas que aseguren la calidad continua del código que permitan evaluar la deuda técnica del proyecto.
	\item Probar la aplicación con ejemplos reales y utilizando usos avanzados de bibliotecas, como la conjuntos de datos entrada de test en ficheros con formato tabulado tipo csv (\textit{comma separated values}). 	
\end{itemize}
