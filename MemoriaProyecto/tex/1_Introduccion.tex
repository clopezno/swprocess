\capitulo{1}{Introducción}

%Descripción del contenido del trabajo y del estrucutra de la memoria y del resto de materiales entregados.

Un proceso software es un conjunto de actividades cuya meta es el desarrollo o evolución del software. Durante el proceso intervienen múltiples factores: el equipo de desarrollo, el tipo de producto software, la estabilidad de los requisitos funcionales, la importancia de los requisitos no funcionales como escalabilidad, seguridad, licencias, lenguaje de programación, tipo de arquitectura de computación, etc. Esto hace que el proceso sea bastante complejo.

Para superar esta complejidad se definen modelos que ayudan a definir las actividades y artefactos del proceso de desarrollo de software. Los artefactos son las salidas de las actividades y el conjunto de artefactos conforman el producto software. En el caso de  Unified Process (UP) \citep{jacobson_proceso_2000} se identifican las siguientes actividades o flujos de trabajo: recolección de requisitos, diseño e implementación, pruebas y despliegue. Además en UP se añaden tres flujos de trabajo de soporte: configuración de cambios, gestión de proyecto y gestión de entorno. Estos flujos de trabajo se aplican iterativamente durante varias fases del desarrollo en cada una de las cuales se incrementa el producto software con algún artefacto resultado de la actividad. La característica de iteración e incremental es recogida en otros métodos o buenas prácticas del desarrollo ágil: Scrum, eXtreme Programming, Lean...

%\todo Añadir referencias bibliográficas a libros

Los repositorios de código son espacios virtuales donde los equipos de desarrollo generan los artefactos colaborativos procedentes de las actividades de un proceso de desarrollo. Además de guardar los artefactos, la versión final y anteriores versiones, estos repositorios, normalmente, permiten almacenar la interacción de los miembros del equipo justificando el cambio de versión. Dependiendo del artefacto generado se utiliza distintos sistemas: foros de comunicación, sistemas de control de versiones, sistemas de gestión de incidencias, sistemas de gestión de pruebas, sistemas de revisiones de calidad, sistemas de integración y despliegue continuo, etc. \citep{guemes-pena_emerging_2018}.

En la última década han surgido forjas de proyectos software de fácil acceso tanto para proyectos empresariales como para proyectos open-source (SourceForge \footnote{\url{https://sourceforge.net/}}, GitHub \footnote{\url{https://github.com/}}, GitLab \footnote{\url{https://about.gitlab.com/}}, Bitbucket  \footnote{\url{https://bitbucket.org/}}).  Estas forjas suelen integrar múltiples sistemas para dar soporte a los flujos de trabajo y registrar las interacciones entre los miembros del equipo. Además dan la posibilidad de extensión funcional con sistemas de terceros para gestionar otras actividades no soportadas directamente por la propia forja, por ejemplo Travis CI \footnote{\url{https://travis-ci.org/}} para gestionar la integración continua o Codacy \footnote{\url{https://www.codacy.com/}} para gestionar las revisiones automáticas de calidad. 

Parece lógico considerar como hipótesis que la calidad de un artefacto software tenga alguna relación con la manera en la que el equipo de desarrollo aplica las actividades del proceso dentro del repositorio. Sommerville expone en \textit{Ingeniería de software} \citep{sommerville_ingenierisoftware_2002} que la calidad del proceso es uno de los factores que afectan a la calidad del producto, junto con las tecnologías utilizadas para el desarrollo, la calidad del personal y el coste, tiempo y duración del proyecto. La validación empírica de estas  hipótesis ha abierto una nueva línea de aplicación con los conjuntos de datos que se pueden extraer de estos repositorios gracias a interfaces de programación específicas que proporcionan estas forjas de repositorios y que permiten acceder a toda la información registrada.

El  desafío a la comunidad científica y empresarial es constante mostrando un incremento en el interés en las aplicaciones que permitan mejorar sus sistemas de decisión. Estas aplicaciones deberán llevar un control sobre el proceso y/o sobre el producto software y ese control se podrá realizar mediante un proceso de medición. La medición puede ser de control o de medición. La primera se asocia al proceso, mientras que la segunda se asocia al producto software. Estas forjas de proyectos software están en constante evolución, tanto en sus estructuras estáticas como en sus interacciones dinámicas en los proyectos. Se registran grandes conjuntos de datos difíciles de procesar y son de estos donde se pueden obtener tantas métricas de control como de predicción.

En este TFG se diseña un software para calcular métricas de control \footnote{También llamadas métricas de proceso o métricas de evolución} sobre distintos repositorios. En el diseño se ha optado por implementar una aplicación web escrita en lenguaje Java que toma como entrada un conjunto de repositorios públicos o privados de GitLab y calcula métricas de evolución que permiten comparar los proyectos. Además, se ha procurado un diseño extensible a otras forjas de repositorios y se ha facilitado la incorporación de nuevas métricas. La aplicación ha sido probada con Trabajos Fin de Grado presentados en la Universidad de Burgos y que han sido almacenados en repositorios públicos de GitLab.

\section{Estructura de la memoria}

La memoria de este trabajo se estructura de la siguiente manera \footnote{\url{https://github.com/ubutfgm/plantillaLatex}} \citep{ubu_plantilla_2019}:

\begin{description}
	\tightlist
	\item[Introducción.] Introducción al trabajo realizado, estructura de la memoria y listado de materiales adjuntos.
	\item[Objetivos del proyecto.] Objetivos que se persiguen alcanzar con la realización del proyecto.
	\item[Conceptos teóricos.] Conceptos clave para comprender los objetivos, el proceso y el producto del proyecto.
	\item[Técnicas y herramientas.] Técnicas y herramientas utilizadas durante el desarrollo del proyecto.
	\item[Aspectos relevantes del desarrollo.] Aspectos destacables durante el proceso de desarrollo del proyecto.
	\item[Trabajos relacionados.] Otros proyectos de la misma naturaleza y los cuales han ayudado a la realización de este.
	\item[Conclusiones y líneas de trabajo futuras.] Conclusiones tras la realización del proyecto y posibilidades de mejora o expansión.
\end{description}

Se incluyen también los siguientes anexos:

\begin{description}
	\tightlist
	\item[Plan del proyecto software.] Planificación temporal y estudio de la viabilidad del proyecto.
	\item[Especificación de requisitos del software.] Análisis de los requisitos.
	\item[Especificación de diseño.] Diseño de los datos, diseño procedimental y diseño arquitectónico.
	\item[Manual del programador.] Aspectos relevantes del código fuente.
	\item[Manual de usuario.] Manual de uso para usuarios que utilicen la aplicación.
\end{description}