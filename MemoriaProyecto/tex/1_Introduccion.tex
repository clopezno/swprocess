\capitulo{1}{Introducción}

%Descripción del contenido del trabajo y del estrucutra de la memoria y del resto de materiales entregados.

 %\imagen{M1-CalidadProcesos}{Calidad basada en procesos}
\begin{quotation}
	\label{q:Sommerville}
	Una suposición subyacente de la administración de la calidad es que la calidad del proceso de desarrollo afecta directamente a la calidad de los productos a entregar \citep[pág 543]{sommerville_ingenierisoftware_2002}.
\end{quotation}
  \imagen{M1-CalidadProcesos}{Aseguramiento de la calidad basada  en procesos}
  
La calidad del proceso es uno de los factores que determinan la calidad del producto software, tal y como expone Sommerville en la figura \ref{fig:M1-FactoresCalidad}.
\imagen{M1-FactoresCalidad}{Principales factores de calidad del producto de software\cite[pág 561]{sommerville_ingenierisoftware_2002}}
Por esa razón, este trabajo se centra en crear un registro de medidas de proceso de proyectos software alojados en GitLab para, posteriormente, evaluarlos en relación con otros proyectos.

\section{Estructura de la memoria}

La memoria se estructura de la siguiente manera\footnote{\url{https://github.com/ubutfgm/plantillaLatex}}\cite{ubu_plantilla_2019}:

\begin{description}
	\tightlist
	\item[Introducción.] Introducción al trabajo realizado, estructura de la memoria y listado de materiales adjuntos.
	\item[Objetivos del proyecto.] Objetivos que se persiguen alcanzar con la realización del proyecto.
	\item[Conceptos teóricos.] Conceptos clave para comprender los objetivos, el proceso y el producto del proyecto.
	\item[Técnicas y herramientas.] Técnicas y herramientas utilizadas durante el desarrollo del proyecto.
	\item[Aspectos relevantes del desarrollo.] Aspectos destacables durante el proceso de desarrollo del proyecto.
	\item[Trabajos relacionados.] Otros proyectos de la misma naturaleza y los cuales han ayudado a la realización de este.
	\item[Conclusiones y líneas de trabajo futuras.] Conclusiones tras la realización del proyecto y posibilidades de mejora o expansión.
\end{description}

Se incluyen también los siguientes anexos:

\begin{description}
	\tightlist
	\item[Plan del proyecto software.] Planificación temporal y estudio de la viabilidad del proyecto.
	\item[Especificación de requisitos del software.] Análisis de los requisitos.
	\item[Especificación de diseño.] Diseño de los datos, diseño procedimental y diseño arquitectónico.
	\item[Manual del programador.] Aspectos relevantes del código fuente.
	\item[Manual de usuario.] Manual de uso para usuarios que utilicen la aplicación.
\end{description}