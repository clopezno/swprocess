\apendice{Plan de Proyecto Software}\label{anex:A}

\section{Introducción}
%Ref david goo bees https://github.com/davidmigloz/go-bees puede ayudar con la memoria
Una de las actividades de un proyecto software es la fase de planificación. En esta fase se estima el esfuerzo, el tiempo y el dinero que se espera invertir en el proyecto. El objetivo principal de la fase de planificación es estimar si se puede realizar el proyecto con éxito y, en ese caso, tener una guía de referencia para el proceso de desarrollo. Este anexo recoge los documentos generados en este proceso y se divide en dos partes:
\begin{description}
	\tightlist
	\item Planificación temporal. En esta parte se estima el tiempo y el esfuerzo que se requieren para la realización del proyecto.
	\item Estudio de la viabilidad. Esta parte se estima si el proyecto es viable tanto económica como legalmente, por tanto se dividirá en dos secciones:
	\begin{description}
		\tightlist
		\item[Viabilidad económica]: Análisis del coste y del beneficio que supondría la realización del proyecto.
		\item[Viabilidad legal]: Análisis de las leyes que se aplicarían desde el comienzo del proyecto. En un proyecto software tienen especial importancia las licencias y la Ley de Protección de Datos.
	\end{description}
\end{description}

\section{Planificación temporal}
No se ha realizado una planificación temporal del proyecto. Se han seguido los 12 principios del manifiesto ágil y el modelo SCRUM \cite{noauthor_scrum_2019}:
\begin{itemize}
	\tightlist
	\item Se ha aplicado un desarrollo incremental y evolutivo.
	\item Se han realizado iteraciones (\textit{sprints}) de dos semanas. Al finalizar un sprint se realizaba una reunión entre el tutor y el alumno que daba comienzo al siguiente sprint y que consta de dos partes:
	\begin{itemize}
		\item Una parte de revisión del sprint en la que se exponía una parte operativa del producto realizada durante el sprint.
		\item Otra de planificación del siguiente sprint en la que se determinaba el trabajo y los objetivos a alcanzar durante el siguiente sprint. Esto quedaba reflejado como una pila de tareas que se debían completar durante el sprint y que han sido registradas en el sistema de gestión de incidencias de GitLab.
	\end{itemize}
\end{itemize}
\subsection{Sprints}
A continuación se definen los sprints y sus respectivas pilas de tareas que se llevaron a cabo durante la realización del proyecto. 
Se pueden visualizar las issues en orden ascendente de creación a partir de esta URL:

\subsubsection{Inicio del proyecto}
El proyecto comenzó el 1 de octubre de 2018. Las primeras tareas que se definieron fueron de investigación y configuración del entorno de trabajo, no se definían de forma muy clara ya que aun no se sabía que caminos escoger para la realización del proyecto.

\section{Estudio de viabilidad}
\subsection{Viabilidad económica}
En esta sección se realiza un análisis coste-beneficio del proyecto.
\subsection{Viabilidad legal}
En esta sección se realiza un estudio de las leyes que se aplican a este proyecto para asegurar que la realización del proyecto no supondrá ninguna violación de estas leyes. 


