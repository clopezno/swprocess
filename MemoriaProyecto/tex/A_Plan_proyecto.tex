\apendice{Plan de Proyecto Software}

\section{Introducción}
%Ref david goo bees https://github.com/davidmigloz/go-bees puede ayudar con la memoria
El plan del proyecto es un documento que recoge el tiempo, esfuerzo y el dinero que supondrá la realización del proyecto.
Este plan se divide en dos partes:
\begin{itemize}
	\tightlist
	\item Planificación temporal
	\item Estudio de la viabilidad
\end{itemize}
El objetivo de la primera es estimar el tiempo y el esfuerzo que se requieren para la realización del proyecto, 
La segunda parte se centra en un análisis de la viabilidad del proyecto. Se objetivo es estimar si el proyecto se podría realizar con éxito. El análisis de viabilidad se puede ha dividido en dos apartados:
\begin{itemize}
	\tightlist
	\item
	\textbf{Viabilidad económica}: Análisis del coste y del beneficio que supondría la realización del proyecto.
	\item
	\textbf{Viabilidad legal}: Análisis de las leyes que se aplicarían desde el comienzo del proyecto.En un proyecto software tienen especial importancia las licencias y la Ley
	de Protección de Datos.
\end{itemize}

\section{Planificación temporal}
No se ha realizado una planificación temporal del proyecto. Pero se podría decir que se han tratado de seguir los 12 principios del manifiesto ágil y el modelo SCRUM \cite{noauthor_scrum_2019}:
\begin{itemize}
	\tightlist
	\item Se ha aplicado un desarrollo incremental y evolutivo.
	\item Se han realizado iteraciones (sprints) de dos semanas. Al finalizar un sprint se realizaba una reunión entre el tutor y el alumno que daba comienzo al siguiente sprint y que consta de dos partes:
	\begin{itemize}
		\item Una parte de revisión del sprint en la que se exponía una parte operativa del producto.
		\item Otra de planificación del siguiente sprint en la que se determinaba el trabajo y los objetivos a alcanzar durante el sprint siguiente. Esto quedaba reflejado como una pila de tareas que se debían completar durante el sprint y que han sido registradas en el sistema de gestión de incidencias de GitLab.
	\end{itemize}
\end{itemize}
\subsection{Sprints}
A continuación se definen los sprints y sus respectivas pilas de tareas que se llevaron a cabo durante la realización del proyecto. 
Se pueden visualizar las issues en orden ascendente de creación a partir de esta URL:

\subsubsection{Inicio del proyecto}
El proyecto comenzó el 1 de octubre de 2018. Las primeras tareas que se definieron fueron de investigación y configuración del entorno de trabajo, no se definían de forma muy clara ya que aun no se sabía que caminos escoger para la realización del proyecto.

\section{Estudio de viabilidad}

\subsection{Viabilidad económica}

\subsection{Viabilidad legal}


