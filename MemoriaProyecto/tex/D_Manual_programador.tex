\apendice{Documentación técnica de programación}

\section{Introducción}
Este documento detalla asuntos técnicos de programación.
\section{Estructura de directorios}
El código fuente presenta la siguiente estructura:
\begin{description}
	\tightlist
	\item[\textit{/.gitignore}] Contiene los ficheros y directorios que el repositorio git no tendrá en cuenta
	\item[\textit{/.gitlab-ci.yml}] Contiene las etapas y trabajos que se han definido para que se ejecuten en una máquina virtual proporcionada por GitLab (\textit{runner}) tras hacer un commit. Permite la integración y el despliegue continuo.
	\item[\textit{/README.md}] Fichero con información relevante sobre el proyecto.
	\item[\textit{/codacy-coverage-reporter-4.0.5-assembly.jar}] Ejecutable Java que necesario para una de las tareas de la integración continua: El informe de cobertura de las pruebas. Se ejecutará en una de las tareas definidas en el fichero /.gitlab-ci.yml.
	\item[\textit{/pom.xml}] Fichero de configuración del proyecto maven.
	\item[\textit{/system.properties}] Fichero con propiedades del proyecto. Ha sido necesario su uso para el despliegue en Heroku.
	\item[\textit{/MemoriaProyecto}] Memoria del proyecto según la plantilla definida en \url{https://github.com/ubutfgm/plantillaLatex}.
	\item[\textit{/src/test/resources}] Datos almacenados en ficheros CSV para proporcionar datos a test parametrizados.
	\item[\textit{/src/test/java}] Casos de prueba JUnit para la realización de pruebas. Se organiza de la misma forma que /src/main/java
	\item[\textit{/src/main/webapp/VAADIN/themes/MyTheme}] Tema principal utilizado por la aplicación. Generado por Vaadin.
	\item[\textit{/src/main/webapp/frontend}] Ficheros \textit{.css} utilizados por la interfaz gráfica.
	\item[\textit{/src/main/webapp/images}] Imágenes que se muestran en la interfaz gráfica.
	\item[\textit{/src/main/resources}] Ficheros de configuración de la aplicación. En este caso el fichero log4j2.properties para configurar el log.
	\item[\textit{/src/main/java}] Contiene todo el código fuente
	\item[\textit{/src/main/java/app/}] Contiene fachadas que conectan la interfaz de usuario con el resto de componentes que componen la lógica de la aplicación.
	\item[\textit{/src/main/java/app/listeners}] Contiene observadores y eventos utilizados por la aplicación
	\item[\textit{/src/main/java/datamodel}] Contiene el modelo de datos de la aplicación.
	\item[\textit{/src/main/java/exceptions}] Contiene excepciones definidas en la aplicación.
	\item[\textit{/src/main/java/gui}] Contiene la interfaz de usuario.
	\item[\textit{/src/main/java/gui/views}] Contiene páginas y componentes de Vaadin que componen la interfaz gráfica de la aplicación.
	\item[/src/main/java/metricsengine] Define el motor de métricas.
	\item[\textit{/src/main/java/metricsengine/numeric\_value\_metrics}] Métricas definidas por el programador y sus respectivas fábricas (Patrón de diseño método fábrica\footnote{https://refactoring.guru/design-patterns/factory-method}). Todas las métricas tienen resultados numéricos.
	\item[\textit{/src/main/java/metricsengine/values}] Valores que devuelven las métricas.
	\item[\textit{/src/main/java/repositorydatasource}] Framework de conexión a una forja de repositorios como GitLab.
\end{description}
\section{Manual del programador}
Se explica en este apartado algunas bases para entender como continuar la programación de la aplicación y los puntos de extensión que se han definido.
Cada apartado de esta sección se centra en cada uno de los módulos que contiene la aplicación.
\subsection{Framework de conexión}
El framework de conexión a una plataforma de desarrollo colaborativo está definido en el paquete \textit{repositorydatasource}. Consta de dos interfaces, la más importante es la interfaz \textit{RepositoryDataSource}.
\subsection{Motor de métricas}
El motor de métricas se ha desarrollado con una base inicial a una solución propuesta en \textit{Soporte de Métricas con Independencia del Lenguaje para la Inferencia de Refactorizaciones}\cite{marticorena_soporte_2005}. El diseño se puede observar en \ref{MCTMotorMetricas}.
\imagen{MCTMotorMetricas}{Diagrama del Framework para el cálculo de métricas con perfiles.}

\subsection{Interfaz gráfica}

\subsection
\section{Compilación, instalación y ejecución del proyecto}

\section{Pruebas del sistema}
Se ha generado una batería de pruebas en \textit{src/test/java}. Algunos de estos test son parametrizados y los valores se encuentran en ficheros \textit{.csv} en la carpeta \textit{src/test/resources}.
